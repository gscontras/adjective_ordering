\documentclass[12pt]{article}
\usepackage[hmargin={.85in},vmargin={.85in}]{geometry}   
\geometry{a4paper}       
%\geometry{landscape}          
%\usepackage[parfill]{parskip}
\usepackage{color,graphicx}
%\usepackage{covington}
%\usepackage{xyling}
\usepackage{setspace}
\usepackage{amsmath}
\usepackage{amssymb}
%\usepackage{graphicx,color}
%\usepackage{theorem}
%\usepackage{tabularx}
%\usepackage{subfig}
%\usepackage{vowel}
%\usepackage{mathrsfs}
\usepackage{varioref}
\usepackage{textcomp}
%\usepackage{avm}
\usepackage{textcomp}
\usepackage{mflogo}
\usepackage{wasysym}
%\usepackage{pstricks, pst-plot, pst-node, pst-tree, colortab}
%\usepackage{qtree}
 %\usepackage{tree-dvips}
 \usepackage{linguex}
%\usepackage{gb4e}
 \usepackage{multirow}
 %\usepackage[stable]{footmisc}
 \usepackage{pifont}
%\usepackage{todonotes}
%\usepackage{natbib}
\usepackage[normalem]{ulem}
\usepackage{wrapfig}

 %\setlength{\parskip}{.55ex plus 0.1ex}


\usepackage{fancyhdr} % This should be set AFTER setting up the page geometry
\pagestyle{plain} % options: empty , plain , fancy
\lhead{}\chead{}\rhead{}
\renewcommand{\headrulewidth}{.3pt}
\lfoot{}\cfoot{\thepage}\rfoot{}
%\renewcommand{\footrulewidth}{.3pt}
\newcommand{\txtp}{\textipa}
\renewcommand{\rm}{\textrm}
\newcommand{\sem}[1]{\mbox{$[\![$#1$]\!]$}}
\newcommand{\lam}{$\lambda$}
\newcommand{\lan}{$\langle$}
\newcommand{\ran}{$\rangle$}
\newcommand{\type}[1]{\ensuremath{\left \langle #1 \right \rangle }}
\newcommand{\defeq}{$\mathrel{\mathop:}=$ }
\renewcommand{\and}{$\wedge$ }


%\renewcommand{\Extopsep}{2pt}


\newcommand{\bex}{\begin{examples}}
\newcommand{\eex}{\end{examples}}

%bullet points
\newcommand{\bit}{\begin{itemize}}
\newcommand{\eit}{\end{itemize}}

%numbering, non sequential
\newcommand{\ben}{\begin{enumerate}}
\newcommand{\een}{\end{enumerate}}

\renewcommand{\abstractname}{The goal:}


%numbering, what you would use in a paper when you don't want the numbering to stop every time you end an example. 
%\newcommand{\bex}{\begin{enumerate}\setcounter{enumi}{\thesaveenumi}\item{}\begin{enumerate}}
%\newcommand{\eex}{\end{enumerate}\setcounter{saveenumi}{\theenumi}\end{enumerate}}

%%these are the brackets used for writing up semantic meanings 
%\newcommand{\lbr}{\textrm{\textlbrackdbl}}
%\newcommand{\rbr}{\textrm{\textrbrackdbl}}
%\renewcommand{\rm}{\textrm}

%this describes the numbering system (roman vs arabic numerals and so forth)
\renewcommand\theenumi {\alph{enumi}}
\renewcommand\theenumii {\alph{enumii}}
\renewcommand\labelenumi {\theenumi. }
\renewcommand\labelenumii {\theenumii.}
\labelformat{enumi}{(\theenumi)}
\labelformat{enumii}{(\theenumi\theenumii)}
\newcounter{saveenumi}

%\renewcommand{\labelitemi}{\textbf{---}}
%\renewcommand{\labelitemii}{\textbf{$\cdot$}}

\linespread{1.5}

%\qtreecenterfalse

%\linespread{1}

\begin{document}
	\thispagestyle{empty}

\begin{center}\textbf{Property subjectivity predicts adjective ordering preferences}\\
	Gregory Scontras, Judith Degen, Noah D.~Goodman\\
	\emph{Stanford University}
\end{center}
	
%	\vspace{-12pt}
	
Cross-linguistically stable preferences for adjective ordering have been widely documented, yet the factors that determine these preferences are still poorly understood. Our approach to the investigation of adjective ordering preferences synthesizes strategies from the \emph{psychological approach}, probing the principles that underlie these preferences, and from the \emph{grammatical approach}, using descriptive semantic classes of adjectives to structure and our investigation and smooth our data. Distilling the psychological proposals that precede us into a single feature, we advance the hypothesis that it is the \emph{subjectivity} of the property named that determines ordering preferences: less subjective adjectives occur closer to the substantive head of the nominal projection, that is, to the modified noun. 
While subjectivity can be assessed directly (by asking participants), we show it can more reliably be measured as the extent to which two people can disagree about a description without one necessarily being wrong (i.e., the degree to which an adjective description admits \emph{faultless disagreement}). 
In ``the big blue box,'' judgments about bigness are likely less consistent than judgments about blueness; ``blue'' is less subjective than ``big,'' and so it occurs closer to the noun ``box.''

To test the hypothesis that adjective subjectivity predicts ordering preferences, we established two empirical constructs: the preferences themselves, which we measured using naturalness ratings and validated with corpus statistics; and adjective subjectivity, which we operationalized as potential for faultless disagreement and corroborated with a direct ``subjectivity'' measure. 
We find that an adjective's semantics does predict its distance from the modified noun: faultless disagreement scores account for 88\% of the variance in our ordering preference data ($r^2$ = 0.88, 95\% CI [0.77, 0.95]). Our results suggest that ordering preferences likely emerge, at least partially, from a desire to place less subjective content closer to the substantive head of a nominal construction (i.e., closer to the modified noun). 
For now we can only speculate about the ultimate source of this desire: Subjective content allows for miscommunication to arise if speakers and listeners arrive at different judgments about a property description. Hence, less subjective content is more useful at communicating about the world. 
Whatever its source, the success of subjectivity in predicting adjective ordering preferences provides a compelling case where linguistic universals, the regularities we observe in adjective ordering, emerge from cognitive universals, the subjectivity of the properties that the adjectives name.




\end{document}




