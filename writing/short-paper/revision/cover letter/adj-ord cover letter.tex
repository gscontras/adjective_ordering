\documentclass[12pt]{article}
\usepackage[hmargin={1in},vmargin={1in,1in},foot={.6in}]{geometry}   
\geometry{letterpaper}              
%\usepackage[parfill]{parskip}
\usepackage{color,graphicx}
\usepackage{setspace}
\usepackage{amsmath}
\usepackage{amssymb}
\usepackage{varioref}
\usepackage{textcomp}
\usepackage{avm}
\usepackage{textcomp}
\usepackage{mflogo}
\usepackage{wasysym}
\usepackage[normalem]{ulem}
\usepackage{hyperref}

\newcommand{\HRule}{\rule{\linewidth}{0.25mm}}

\usepackage{fancyhdr} % This should be set AFTER setting up the page geometry
\pagestyle{plain} % options: empty , plain , fancy
\lhead{}\chead{}\rhead{}
\renewcommand{\headrulewidth}{.5pt}
\lfoot{}\cfoot{\thepage}\rfoot{}
\newcommand{\txtp}{\textipa}
\renewcommand{\rm}{\textrm}
\newcommand{\sem}[1]{\mbox{$[\![$#1$]\!]$}}
\newcommand{\lam}{$\lambda$}
\newcommand{\lan}{$\langle$}
\newcommand{\ran}{$\rangle$}
\newcommand{\type}[1]{\ensuremath{\left \langle #1 \right \rangle }}

\newcommand{\bex}{\begin{exe}}
\newcommand{\eex}{\end{exe}}
\newcommand{\bit}{\begin{itemize}}
\newcommand{\eit}{\end{itemize}}
\newcommand{\ben}{\begin{enumerate}}
\newcommand{\een}{\end{enumerate}}

%\linespread{1.5}
\thispagestyle{plain}

\begin{document}

{\flushright

\vspace{25pt}
Gregory Scontras\\
Judith Degen\\
Noah D.~Goodman\\
Department of Psychology\\
Stanford University\\
Stanford, CA 94305\\[20pt]

\noindent October XXX, 2015\\[20pt]}


\noindent Dear Editor,\\

\noindent We would like to thank you and the two anonymous reviewers for the helpful reviews of our paper, ``Subjectivity predicts adjective ordering preferences.'' As you will recall, you and Reviewer 2 were in agreement that adjective ordering would be an excellent topic for a PNAS paper. However, both you and the reviewers expressed concern that our work does not move the inquiry forward far enough. We believe this concern stems from a failure on our part to do justice to the significant empirical contributions our paper makes.

Adjective ordering preferences continue to recur in discussions of language universals precisely because of their reported regularity within and across languages. However, even in the best studied case---English---the investigations to date have been largely impressionistic rather than empirical. Most authors report their own intuitions, or the intuitions of a handful of informants. The only large-scale empirical studies of ordering preferences that we know of are Martin 1969, who limits himself to behavioral measures (i.e., speaker intuitions), and Wulff 2013, who limits herself to corpus measures (i.e., relative frequencies). Our paper marries these two approaches, testing one against the other in order to arrive at clear estimates of the  preferences themselves.

If past reporting of adjective ordering preferences is impressionistic, then the discussion of the factors that contribute to them is metaphysical.

In the remainder of this letter, we discuss in more detail how we responded to each of the reviewers' concerns.


\subsubsection*{Reviewer 1:}

\ben

\item \emph{The fundamental factor in predicting adjective ordering is whether an adjective is used to form a complex concept/subkind description or not.}

Response.

\item  \emph{Subkind descriptions could be a confounding factor in the corpus study if color or material terms are more frequently used to create subkind descriptions.}

Response.

\item \emph{Other known factors that affect adjective ordering like contrastiveness in discourse should be discussed.}

Response.

\item \emph{The finding that subjectivity predicts adjective ordering preferences lacks an explanation.}

Response.

\item \emph{Linguists were already aware of subjectivity in adjectives.}

Response.

\item \emph{How was the number of participants chosen?}

Response.

\item \emph{Was any effort made to control how often adjectives appeared with other adjectives in the ordering experiment?}

Response

\een




\subsubsection*{Reviewer 2:}

\ben

\item \emph{There are two notions of faultless disagreement: 1) concerning semantic content, and 2) concerning context sensitivity and perspective. It was not clear which notion we intended.}

Response

\item \emph{If successful referential communication is the driving factor, then we would predict the subjectivity gradient for post-nominal adjectives, but not for pre-nominal adjectives.}

Response.

\item \emph{What about numerals, which some people take to be adjectives?}

Response.

\een




\noindent Thank you again for the comments on our work. We hope that you all like the new version of the paper. Please let us know if you require additional information. We look forward to hearing from you!\\[25pt]


\noindent Yours sincerely,\\[10pt]

\noindent Gregory Scontras, Judith Degen, and Noah D.~Goodman



\end{document}














