\documentclass[12pt]{article}
\usepackage[hmargin={1in},vmargin={1in,1in},foot={.6in}]{geometry}   
\geometry{letterpaper}              
\usepackage{color,graphicx}
\usepackage{setspace}
\usepackage{amsmath}
\usepackage{amssymb}
\usepackage{varioref}
\usepackage{textcomp}
\usepackage{textcomp}
\usepackage{mflogo}
\usepackage{wasysym}
\usepackage[normalem]{ulem}
\usepackage{hyperref}

\newcommand{\HRule}{\rule{\linewidth}{0.25mm}}

\usepackage{fancyhdr} % This should be set AFTER setting up the page geometry
\pagestyle{plain} % options: empty , plain , fancy
\lhead{}\chead{}\rhead{}
\renewcommand{\headrulewidth}{.5pt}
\lfoot{}\cfoot{\thepage}\rfoot{}
\newcommand{\txtp}{\textipa}
\renewcommand{\rm}{\textrm}
\newcommand{\sem}[1]{\mbox{$[\![$#1$]\!]$}}
\newcommand{\lam}{$\lambda$}
\newcommand{\lan}{$\langle$}
\newcommand{\ran}{$\rangle$}
\newcommand{\type}[1]{\ensuremath{\left \langle #1 \right \rangle }}

\newcommand{\bex}{\begin{exe}}
\newcommand{\eex}{\end{exe}}
\newcommand{\bit}{\begin{itemize}}
\newcommand{\eit}{\end{itemize}}
\newcommand{\ben}{\begin{enumerate}}
\newcommand{\een}{\end{enumerate}}

\newcommand{\gcs}[1]{\textcolor{blue}{[gcs: #1]}}
\definecolor{Green}{RGB}{10,200,100}
\newcommand{\ndg}[1]{\textcolor{Green}{[ndg: #1]}}
\newcommand{\jd}[1]{\textcolor{red}{[jd: #1]}}

\thispagestyle{empty}

\begin{document}

{\flushright

\vspace{25pt}
Irvine, California\\[20pt]

\noindent July XXX, 2018\\[25pt]}


\noindent Dear Dr.~S\ae b\o,\\

\noindent We would like to thank you and the anonymous reviewer for your comments on our resubmitted paper, ``On the grammatical source of adjective ordering preferences.'' As you will recall, you identified a relatively clear path to publication on the basis of addressing the following concerns:

\ben

\item \emph{My remaining problem is with the way you address the third of the three
	concerns I highlighted in my letter of decision on last year's submission,
	namely, that it should be shown more explicitly how starting the composition
	with less noisy modifiers leads to a lower risk of misclassification and a
	greater chance of successful reference resolution than starting with more
	noisy modifiers. You now lay out in more detail, in the middle of page 10,
	how starting with less noisy modifiers indeed yields a lower probability of
	there being at least one misclassification. It took me some time to see it,
	however, and it would be helpful if you could supply what I requested in my
	letter, a step-by-step demonstration that ``small brown box" results in fewer
	errors, or more accurately a higher probability of no error, than ``brown
	small box", complete with (toy) example numbers of potential referents and
	probabilities. It will also ease understanding if you clarify and simplify
	the introduction of the noise measure as the probability of
	misclassification; as far as I can see, what it says between ``..., we
	introduce noise into the semantics of our adjectives" and ``For each
	potential referent an adjective classifies, ..." is superfluous and a
	possible source of confusion (I wrote that the definition (3) is not
	immediately intelligible, and it is not more intelligible now; I suggest you
	delete it).}
	
We apologize for making you repeat yourself; we have added a step-by-step demonstration calculating the probability of misclassications for \emph{small brown box} vs.~\emph{brown small box}. We have also moved what was the definition in (3), together with a short explanation, to footnote 1. We chose to keep the definition in a footnote for those readers who are interested in how one could formalize potential for misclassification in a truth-functional semantics.
	
\item \emph{In a next step, you set out to show how starting with less noisy modifiers
	leads to a greater chance of successful reference resolution. Again, and
	more critically, a step-by-step demonstration that ``small brown box" raises
	the probability of successful reference resolution vis-\`{a}-vis ``brown small
	box", once ``small" is analyzed as a subsective modifier that maps the
	meaning of ``(brown) box" to the set of (brown) boxes that are small for
	(brown) boxes, is missing. The third paragraph on page 11 does not make it
	clear how under that analysis, there is a higher probability of correctly
	classifying the intended referent if ``small" combines with ``brown box" than
	if ``brown" combines with ``small box". So again, a toy example with concrete
	estimates would be helpful. There is a second option here which I would ask you to consider. Note that
	the reviewer takes a critical view of explaining the ordering preference in
	terms of reference resolution and intended referent retrieval. In light of
	that, if it turns out to be difficult to show clearly that a
	non-intersective semantics for adjectives like ``small" is necessary and
	sufficient to derive that ``small brown box" maximizes the probability of
	successful reference resolution, i.e., arriving at the correct referent, it
	might be a good idea to leave that part out and concentrate the argument on
	the heightened probability of avoiding any misclassification if relatively
	noisy modifiers enter into the semantic composition at a relatively `late'
	stage. So please give consideration to this option.}

We are particularly grateful to you for raising this concern, as we believe addressing it has substantially clarified our claims and strengthened our argument. Previously, we equivocated on the mechanism whereby subsective modification breaks the commutativity of noise. We did this in order to avoid unnecessarily distracting (or contentious) assumptions, but the result, as you rightly point out, was an unnecessarily distracting and contentious lack of clarity. Now, we provide the requested step-by-step demonstration for \emph{small brown box} vs.~\emph{brown small box}, as requested, making the minimal assumptions necessary along the way. Given that this demonstration and the one mentioned in our response to your first point feature a good deal of math not common to the average linguistics article, we have set these demonstrations apart in their own subsection, advising readers who are already convinced by our argument or wary of math that the demonstrations can be skipped.
	
	
\een

Thank you again for the thorough and thoughtful comments on our work. We hope that you will like the new version of the paper. Please let us know if you require additional information. We look forward to hearing from you!\\[25pt]


\noindent Yours sincerely,\\[10pt]

\noindent Gregory Scontras, Judith Degen, and Noah D.~Goodman

\newpage





\end{document}














